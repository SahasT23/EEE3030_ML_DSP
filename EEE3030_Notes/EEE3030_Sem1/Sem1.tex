\documentclass[a4paper,12pt]{article}

%----------------------------------------------------------------------------------------
%	FONT
%----------------------------------------------------------------------------------------

% % fontspec allows you to use TTF/OTF fonts directly
% \usepackage{fontspec}
% \defaultfontfeatures{Ligatures=TeX}

% % modified for ShareLaTeX use
% \setmainfont[
% SmallCapsFont = Fontin-SmallCaps.otf,
% BoldFont = Fontin-Bold.otf,
% ItalicFont = Fontin-Italic.otf
% ]
% {Fontin.otf}

%----------------------------------------------------------------------------------------
%	PACKAGES
%----------------------------------------------------------------------------------------
\usepackage{url}
\usepackage{parskip} 	
\usepackage{amsmath}
\usepackage{amssymb}
\usepackage{amsthm}

\usepackage{pgfplots}
\usepackage{caption}
\usepackage{subcaption}
\usepackage{tikz}
\usetikzlibrary{arrows.meta, calc, positioning}

%other packages for formatting
\RequirePackage{color}
\RequirePackage{graphicx}
\usepackage[scale=0.9]{geometry}

%tabularx environment
\usepackage{tabularx}

%for lists within experience section
\usepackage{enumitem}

% centered version of 'X' col. type
\newcolumntype{C}{>{\centering\arraybackslash}X} 

%to prevent spillover of tabular into next pages
\usepackage{supertabular}
\usepackage{tabularx}
\newlength{\fullcollw}
\setlength{\fullcollw}{0.47\textwidth}

%custom \section
\usepackage{titlesec}				
\usepackage{multicol}
\usepackage{multirow}

%CV Sections inspired by: 
%http://stefano.italians.nl/archives/26
\titleformat{\section}{\large\scshape\raggedright}{}{0em}{}[\titlerule]
\titlespacing{\section}{0pt}{10pt}{10pt}

%for publications
\usepackage[style=authoryear,sorting=ynt, maxbibnames=2]{biblatex}

%Setup hyperref package, and colours for links
\usepackage[unicode, draft=false]{hyperref}
\definecolor{linkcolour}{rgb}{0,0.2,0.6}
\hypersetup{colorlinks,breaklinks,urlcolor=linkcolour,linkcolor=linkcolour}
\addbibresource{citations.bib}
\setlength\bibitemsep{1em}

%for social icons
\usepackage{fontawesome5}

%debug page outer frames
%\usepackage{showframe}

%----------------------------------------------------------------------------------------
%	BEGIN DOCUMENT
%----------------------------------------------------------------------------------------
\begin{document}

% non-numbered pages
\pagestyle{empty} 

\begin{titlepage}
    \centering
    \vspace*{2cm}
    \Huge{\textbf{EEE3030 Digital Signal Processing Notes}} \\[1.5cm]
    \vfill
\end{titlepage}

% Table of Contents
\tableofcontents
\newpage

\section{\Large\textbf{Introduction}}

This set of notes will cover 1. The prerequisites for learning about digital signal processing methods, 2. The aforementioned methods and how we use them and where they are used. 

\section{\Huge\textbf{Vectors}}

\subsection{What is a Vector?}
A \textbf{vector} is a mathematical object that has both \textbf{magnitude} (length) and \textbf{direction}. 
Vectors are often written in boldface, such as $\mathbf{v}$, or with an arrow, $\vec{v}$. 
In coordinate form, a vector in $\mathbb{R}^n$ is written as
\[
\mathbf{v} = \begin{bmatrix} v_1 \\ v_2 \\ \vdots \\ v_n \end{bmatrix}.
\]

\subsection{Vector Equality}
Two vectors $\mathbf{u}$ and $\mathbf{v}$ are equal if and only if all their components are equal:
\[
\mathbf{u} = \mathbf{v} \quad \Leftrightarrow \quad u_i = v_i \;\; \forall i.
\]

\subsection{Vector Addition}
The sum of two vectors is obtained by adding their corresponding components:
\[
\mathbf{u} + \mathbf{v} = \begin{bmatrix} u_1 \\ u_2 \\ \vdots \\ u_n \end{bmatrix}
+ \begin{bmatrix} v_1 \\ v_2 \\ \vdots \\ v_n \end{bmatrix}
= \begin{bmatrix} u_1 + v_1 \\ u_2 + v_2 \\ \vdots \\ u_n + v_n \end{bmatrix}.
\]

\subsection{Scalar Multiplication}
Multiplying a vector by a scalar $c \in \mathbb{R}$ stretches or shrinks it:
\[
c\mathbf{v} = c \begin{bmatrix} v_1 \\ v_2 \\ \vdots \\ v_n \end{bmatrix}
= \begin{bmatrix} cv_1 \\ cv_2 \\ \vdots \\ cv_n \end{bmatrix}.
\]

\subsection{Vector Magnitude (Norm)}
The length (or magnitude) of a vector $\mathbf{v}$ is
\[
\|\mathbf{v}\| = \sqrt{v_1^2 + v_2^2 + \cdots + v_n^2}.
\]

\subsection{Unit Vectors}
A \textbf{unit vector} has magnitude $1$. To normalize a vector $\mathbf{v}$:
\[
\hat{\mathbf{v}} = \frac{\mathbf{v}}{\|\mathbf{v}\|}.
\]

\subsection{Dot Product (Scalar Product)}
For vectors $\mathbf{u}, \mathbf{v} \in \mathbb{R}^n$:
\[
\mathbf{u} \cdot \mathbf{v} = \sum_{i=1}^n u_i v_i = \|\mathbf{u}\| \|\mathbf{v}\| \cos\theta,
\]
where $\theta$ is the angle between $\mathbf{u}$ and $\mathbf{v}$.
\begin{itemize}
    \item If $\mathbf{u} \cdot \mathbf{v} = 0$, the vectors are orthogonal.
\end{itemize}

\subsection{Cross Product (in $\mathbb{R}^3$)}
For vectors $\mathbf{u}, \mathbf{v} \in \mathbb{R}^3$:
\[
\mathbf{u} \times \mathbf{v} = 
\begin{vmatrix}
\mathbf{i} & \mathbf{j} & \mathbf{k} \\
u_1 & u_2 & u_3 \\
v_1 & v_2 & v_3
\end{vmatrix}
= (u_2v_3 - u_3v_2)\mathbf{i} - (u_1v_3 - u_3v_1)\mathbf{j} + (u_1v_2 - u_2v_1)\mathbf{k}.
\]
The cross product is a vector perpendicular to both $\mathbf{u}$ and $\mathbf{v}$.

\subsection{Projection of One Vector onto Another}
The projection of $\mathbf{u}$ onto $\mathbf{v}$ is
\[
\text{proj}_{\mathbf{v}} \mathbf{u} = 
\left( \frac{\mathbf{u} \cdot \mathbf{v}}{\|\mathbf{v}\|^2} \right) \mathbf{v}.
\]

\subsection{Orthogonality}
Vectors are \textbf{orthogonal} (perpendicular) if
\[
\mathbf{u} \cdot \mathbf{v} = 0.
\]

\subsection{Linear Combination}
A vector $\mathbf{w}$ is a linear combination of $\mathbf{v}_1, \mathbf{v}_2, \dots, \mathbf{v}_k$ if
\[
\mathbf{w} = c_1 \mathbf{v}_1 + c_2 \mathbf{v}_2 + \cdots + c_k \mathbf{v}_k
\]
for some scalars $c_1, c_2, \dots, c_k$.

\subsection{Geometric Visualization}
\begin{center}
\begin{tikzpicture}[scale=1]
\draw[->] (0,0) -- (3,2) node[midway, above] {$\mathbf{u}$};
\draw[->] (0,0) -- (2,3) node[midway, left] {$\mathbf{v}$};
\draw[->, thick, red] (0,0) -- (5,5) node[midway, below right] {$\mathbf{u}+\mathbf{v}$};
\end{tikzpicture}
\end{center}

\section{\Huge\textbf{Matrices}}
A \textbf{matrix} is a rectangular array of numbers arranged in rows and columns.  
An $m \times n$ matrix has $m$ rows and $n$ columns:
\[
A = \begin{bmatrix}
a_{11} & a_{12} & \cdots & a_{1n} \\
a_{21} & a_{22} & \cdots & a_{2n} \\
\vdots & \vdots & \ddots & \vdots \\
a_{m1} & a_{m2} & \cdots & a_{mn}
\end{bmatrix}.
\]

\subsection{Matrix Equality}
Two matrices $A$ and $B$ are equal if they have the same size and
\[
a_{ij} = b_{ij} \quad \forall i,j.
\]

\subsection{Matrix Addition}
If $A, B \in \mathbb{R}^{m \times n}$, then
\[
A + B = \begin{bmatrix}
a_{11} + b_{11} & \cdots & a_{1n} + b_{1n} \\
\vdots & \ddots & \vdots \\
a_{m1} + b_{m1} & \cdots & a_{mn} + b_{mn}
\end{bmatrix}.
\]

\subsection{Scalar Multiplication}
For $c \in \mathbb{R}$,
\[
cA = \begin{bmatrix}
ca_{11} & ca_{12} & \cdots & ca_{1n} \\
\vdots & \vdots & \ddots & \vdots \\
ca_{m1} & ca_{m2} & \cdots & ca_{mn}
\end{bmatrix}.
\]

\subsection{Matrix Multiplication}
If $A \in \mathbb{R}^{m \times n}$ and $B \in \mathbb{R}^{n \times p}$, then
\[
C = AB \in \mathbb{R}^{m \times p},
\]
where each entry is computed as
\[
c_{ij} = \sum_{k=1}^n a_{ik} b_{kj}.
\]

\subsubsection{Example}
\[
\begin{bmatrix}
1 & 2 \\
3 & 4
\end{bmatrix}
\begin{bmatrix}
5 & 6 \\
7 & 8
\end{bmatrix}
=
\begin{bmatrix}
1\cdot 5 + 2\cdot 7 & 1\cdot 6 + 2\cdot 8 \\
3\cdot 5 + 4\cdot 7 & 3\cdot 6 + 4\cdot 8
\end{bmatrix}
=
\begin{bmatrix}
19 & 22 \\
43 & 50
\end{bmatrix}.
\]

\subsection{Transpose}
The \textbf{transpose} of $A \in \mathbb{R}^{m \times n}$ is $A^T \in \mathbb{R}^{n \times m}$, defined by
\[
(A^T)_{ij} = a_{ji}.
\]

\subsection{Identity Matrix}
The identity matrix $I_n$ is an $n \times n$ matrix with ones on the diagonal and zeros elsewhere:
\[
I_3 =
\begin{bmatrix}
1 & 0 & 0 \\
0 & 1 & 0 \\
0 & 0 & 1
\end{bmatrix}.
\]
It satisfies $AI_n = I_m A = A$ for compatible $A$.

\subsection{Determinant}
For a square matrix $A \in \mathbb{R}^{n \times n}$, the \textbf{determinant} $\det(A)$ is a scalar with geometric meaning (volume scaling).  

For $2 \times 2$ matrices:
\[
\det\begin{bmatrix}
a & b \\
c & d
\end{bmatrix} = ad - bc.
\]

For $3 \times 3$ matrices:
\[
\det\begin{bmatrix}
a & b & c \\
d & e & f \\
g & h & i
\end{bmatrix}
= a(ei - fh) - b(di - fg) + c(dh - eg).
\]

\subsection{Inverse of a Matrix}
A square matrix $A$ is invertible if there exists $A^{-1}$ such that
\[
AA^{-1} = A^{-1}A = I.
\]

For $2 \times 2$ matrices:
\[
A = \begin{bmatrix} a & b \\ c & d \end{bmatrix}, \quad
A^{-1} = \frac{1}{ad - bc} \begin{bmatrix} d & -b \\ -c & a \end{bmatrix}, \quad \det(A) \neq 0.
\]

\subsection{Rank of a Matrix}
The \textbf{rank} of a matrix $A$ is the dimension of its column space (or row space).  
It equals the maximum number of linearly independent rows or columns.

\subsection{Eigenvalues and Eigenvectors}
For a square matrix $A$, a nonzero vector $\mathbf{v}$ is an eigenvector if
\[
A\mathbf{v} = \lambda \mathbf{v},
\]
where $\lambda$ is the corresponding eigenvalue.  

To find $\lambda$, solve the characteristic equation:
\[
\det(A - \lambda I) = 0.
\]

\subsection{Geometric Interpretation}
\begin{itemize}
    \item Matrix multiplication can be seen as a linear transformation of space.  
    \item Determinant measures area/volume scaling and orientation.  
    \item Eigenvectors are directions that remain unchanged under the transformation.  
\end{itemize}

\begin{center}
\begin{tikzpicture}[scale=1]
\draw[->] (0,0) -- (2,0) node[midway, below] {$\mathbf{e}_1$};
\draw[->] (0,0) -- (0,2) node[midway, left] {$\mathbf{e}_2$};
\draw[->, red, thick] (0,0) -- (3,1) node[midway, above] {$A\mathbf{e}_1$};
\draw[->, red, thick] (0,0) -- (1,3) node[midway, right] {$A\mathbf{e}_2$};
\end{tikzpicture}
\end{center}

\section{\Huge\textbf{Complex Numbers}}
A complex number is defined as
\[
z = x + iy, \quad x,y \in \mathbb{R}, \quad i^2 = -1.
\]
\[
\Re(z) = x, \quad \Im(z) = y, \quad \overline{z} = x - iy.
\]
\[
|z| = \sqrt{x^2+y^2}, \quad \arg(z) = \theta.
\]

\subsection{Forms}
\[
z = x+iy \quad (\text{rectangular}), \quad
z = r(\cos\theta + i\sin\theta) \quad (\text{polar}), \quad
z = re^{i\theta} \quad (\text{exponential}).
\]

\subsection{Euler’s Formula}
\[
e^{i\theta} = \cos\theta + i\sin\theta.
\]
Special cases:
\[
e^{i\pi} + 1 = 0, \quad e^{i\pi/2} = i.
\]

\subsection{De Moivre’s Theorem}
\[
(re^{i\theta})^n = r^n e^{in\theta} = r^n(\cos n\theta + i \sin n\theta).
\]

\subsection{Argand Diagram}
A geometric representation of $z = x + iy$ as a point $(x,y)$.

\begin{center}
\begin{tikzpicture}[scale=1]
% Axes
\draw[->] (-3,0) -- (3,0) node[right] {Re};
\draw[->] (0,-3) -- (0,3) node[above] {Im};

% Point z
\fill (2,1.5) circle (2pt) node[right] {$z = x+iy$};
\draw[dashed] (2,0) -- (2,1.5) -- (0,1.5);
\draw[->, thick] (0,0) -- (2,1.5) node[midway, above left] {$r=|z|$};

% Angle
\draw (1,0) arc[start angle=0, end angle=37, radius=1] node[midway,right] {$\theta$};
\end{tikzpicture}
\end{center}

\subsection{Phasor Diagram}
A sinusoid
\[
v(t) = V_0 \cos(\omega t + \phi) = \Re\{V_0 e^{i(\omega t + \phi)}\}
\]
is represented by a rotating phasor vector.

\begin{center}
\begin{tikzpicture}[scale=1.2]
% Axes
\draw[->] (-0.5,0) -- (3,0) node[right] {Re};
\draw[->] (0,-0.5) -- (0,2.5) node[above] {Im};

% Phasor
\draw[->, thick, red] (0,0) -- (2,1.2) node[midway, above] {$V_0 e^{i\phi}$};

% Angle phi
\draw (1,0) arc[start angle=0, end angle=31, radius=1] node[midway,right] {$\phi$};
\end{tikzpicture}
\end{center}

\subsection{Exponential\-Trigonometric Relations}
\[
\cos\theta = \frac{e^{i\theta} + e^{-i\theta}}{2}, \quad
\sin\theta = \frac{e^{i\theta} - e^{-i\theta}}{2i}.
\]

\subsection{Hyperbolic Functions}
\[
\cosh x = \frac{e^x + e^{-x}}{2}, \quad \sinh x = \frac{e^x - e^{-x}}{2}.
\]
Relations:
\[
\cos(ix) = \cosh x, \quad \sin(ix) = i\sinh x.
\]


\subsection{Trigonometric Identities}

\subsubsection{Pythagorean}
\[
\sin^2\theta + \cos^2\theta = 1, \quad
1 + \tan^2\theta = \sec^2\theta, \quad
1 + \cot^2\theta = \csc^2\theta.
\]

\subsubsection{Sum and Difference}
\[
\sin(a \pm b) = \sin a \cos b \pm \cos a \sin b,
\]
\[
\cos(a \pm b) = \cos a \cos b \mp \sin a \sin b,
\]
\[
\tan(a \pm b) = \frac{\tan a \pm \tan b}{1 \mp \tan a \tan b}.
\]

\subsubsection{Double Angle}
\[
\sin(2\theta) = 2\sin\theta \cos\theta, \quad
\cos(2\theta) = \cos^2\theta - \sin^2\theta,
\]
\[
\tan(2\theta) = \frac{2\tan\theta}{1 - \tan^2\theta}.
\]

\subsubsection{Half Angle}
\[
\sin^2\frac{\theta}{2} = \frac{1-\cos\theta}{2}, \quad
\cos^2\frac{\theta}{2} = \frac{1+\cos\theta}{2}.
\]

\subsubsection{Product-to-Sum}
\[
\sin a \sin b = \tfrac{1}{2}[\cos(a-b) - \cos(a+b)],
\]
\[
\cos a \cos b = \tfrac{1}{2}[\cos(a-b) + \cos(a+b)],
\]
\[
\sin a \cos b = \tfrac{1}{2}[\sin(a+b) + \sin(a-b)].
\]

\subsubsection{Sum-to-Product}
\[
\sin a \pm \sin b = 2 \sin\left(\tfrac{a \pm b}{2}\right)\cos\left(\tfrac{a \mp b}{2}\right),
\]
\[
\cos a + \cos b = 2 \cos\left(\tfrac{a+b}{2}\right)\cos\left(\tfrac{a-b}{2}\right),
\]
\[
\cos a - \cos b = -2 \sin\left(\tfrac{a+b}{2}\right)\sin\left(\tfrac{a-b}{2}\right).
\]


\subsection{Hyperbolic Identities}

\subsubsection{Fundamental}
\[
\cosh^2 x - \sinh^2 x = 1.
\]

\subsubsection{Sum and Difference}
\[
\sinh(a \pm b) = \sinh a \cosh b \pm \cosh a \sinh b,
\]
\[
\cosh(a \pm b) = \cosh a \cosh b \pm \sinh a \sinh b.
\]

\subsubsection{Double Angle}
\[
\sinh(2x) = 2\sinh x \cosh x, \quad
\cosh(2x) = \cosh^2 x + \sinh^2 x.
\]

\subsubsection{Half Angle}
\[
\cosh^2\frac{x}{2} = \frac{\cosh x + 1}{2}, \quad
\sinh^2\frac{x}{2} = \frac{\cosh x - 1}{2}.
\]

\subsection{Roots of Unity}
The $n$-th roots of unity are the solutions of
\[
z^n = 1.
\]
They are given by
\[
z_k = e^{i \frac{2\pi k}{n}} = \cos\left(\frac{2\pi k}{n}\right) + i\sin\left(\frac{2\pi k}{n}\right),
\quad k = 0,1,2,\dots,n-1.
\]

These correspond to $n$ equally spaced points on the unit circle in the Argand plane.

\begin{center}
\begin{tikzpicture}[scale=2]
    % Unit circle
    \draw[thick] (0,0) circle(1);
    \draw[->] (-1.2,0) -- (1.4,0) node[right] {Re};
    \draw[->] (0,-1.2) -- (0,1.4) node[above] {Im};

    % Roots of unity example for n=6
    \foreach \k in {0,...,5} {
        \coordinate (Z\k) at ({cos(60*\k)},{sin(60*\k)});
        \fill (Z\k) circle (1pt);
        \draw[->, blue] (0,0) -- (Z\k);
        \node[above right] at (Z\k) {$z_{\k}$};
    }

    % Center
    \fill (0,0) circle(0.5pt) node[below left] {0};
\end{tikzpicture}
\end{center}

For example, for $n=6$ the roots are
\[
z_k = e^{i\frac{2\pi k}{6}}, \quad k=0,1,2,3,4,5,
\]
which are vertices of a regular hexagon inscribed in the unit circle.

\subsection{Poles and Zeroes}

In complex analysis and systems theory, the behaviour of a function
\[
F(s) = \frac{N(s)}{D(s)}
\]
is characterized by its \textbf{zeroes} and \textbf{poles}.
\begin{itemize}
    \item A \textbf{zero} is a value $s_0$ such that $F(s_0) = 0$ (i.e. $N(s_0) = 0$).
    \item A \textbf{pole} is a value $s_p$ where $F(s)$ tends to infinity (i.e. $D(s_p) = 0$).
\end{itemize}

\subsubsection{Example:}
Consider
\[
F(s) = \frac{(s-1)(s+2)}{(s-0.5)(s+1)}.
\]
\[
\text{Zeroes: } s=1, \, s=-2 \qquad
\text{Poles: } s=0.5, \, s=-1
\]

\subsubsection{Pole–Zero Diagram}
Poles and zeroes are plotted in the complex plane (often the $s$-plane or $z$-plane):
\begin{itemize}
    \item Zeroes are marked with a $\circ$ (circle).
    \item Poles are marked with a $\times$ (cross).
\end{itemize}

\begin{center}
\begin{tikzpicture}[scale=1.2]
    % Axes
    \draw[->] (-3,0) -- (3,0) node[right] {Re};
    \draw[->] (0,-2) -- (0,2) node[above] {Im};

    % Zeroes (circles)
    \draw (1,0) circle(3pt);
    \node[below right] at (1,0) {$z=1$};
    \draw (-2,0) circle(3pt);
    \node[below left] at (-2,0) {$z=-2$};

    % Poles (crosses)
    \draw (0.5,0.0) node {\Large $\times$};
    \node[above right] at (0.5,0) {$p=0.5$};
    \draw (-1,0.0) node {\Large $\times$};
    \node[above left] at (-1,0) {$p=-1$};

    % Origin
    \fill (0,0) circle(1pt) node[below left] {0};
\end{tikzpicture}
\end{center}

\subsubsection{Interpretation}
\begin{itemize}
    \item Zeroes indicate where the system response is \textbf{cancelled}.
    \item Poles indicate where the system response is \textbf{resonant} or tends to infinity.
    \item The relative locations of poles and zeroes determine the stability and frequency response of a system.
\end{itemize}

\section{\Huge\textbf{Fourier Series}}

Any periodic function $f(t)$ with period $T$ can be expanded into a Fourier series:
\[
f(t) = \frac{a_0}{2} + \sum_{n=1}^{\infty} \left( a_n \cos\left(\frac{2\pi n}{T}t\right) + b_n \sin\left(\frac{2\pi n}{T}t\right) \right).
\]

\subsection{Fourier Coefficients}
\[
a_0 = \frac{2}{T} \int_0^T f(t)\,dt, \quad
a_n = \frac{2}{T} \int_0^T f(t)\cos\left(\tfrac{2\pi n}{T}t\right)\,dt,
\]
\[
b_n = \frac{2}{T} \int_0^T f(t)\sin\left(\tfrac{2\pi n}{T}t\right)\,dt.
\]

\subsection{Example: Square Wave}

Consider a square wave $f(t)$ with period $T = 2\pi$ defined as
\[
f(t) =
\begin{cases}
1, & 0 < t < \pi, \\
-1, & -\pi < t < 0,
\end{cases}
\quad \text{and extended periodically.}
\]

\begin{center}
\begin{tikzpicture}[scale=0.9]
  % Axes
  \draw[->] (-4,0) -- (7,0) node[right] {$t$};
  \draw[->] (0,-1.5) -- (0,1.8) node[above] {$f(t)$};

  % Horizontal dashed lines at y=1 and y=-1
  \draw[dashed] (-4,1) -- (7,1);
  \draw[dashed] (-4,-1) -- (7,-1);

  % Square wave (period 2π)
  \draw[thick] (-4,1) -- (-3.1416,1) -- (-3.1416,-1) -- (0,-1) -- (0,1) -- (3.1416,1) -- (3.1416,-1) -- (6.2832,-1) -- (6.2832,1) -- (7,1);

  % Vertical jumps (dashed)
  \draw[dashed] (-3.1416,-1) -- (-3.1416,1);
  \draw[dashed] (0,-1) -- (0,1);
  \draw[dashed] (3.1416,-1) -- (3.1416,1);
  \draw[dashed] (6.2832,-1) -- (6.2832,1);

  % Labels
  \node at (-3.1416,-1.3) {$-\pi$};
  \node at (0,-1.3) {$0$};
  \node at (3.1416,-1.3) {$\pi$};
  \node at (6.2832,-1.3) {$2\pi$};

  % y-axis labels
  \node[left] at (0,1) {$1$};
  \node[left] at (0,-1) {$-1$};
\end{tikzpicture}
\end{center}


\subsection{Step 1: Compute $a_0$}
\[
a_0 = \frac{2}{2\pi}\int_{-\pi}^{\pi} f(t)\,dt = \frac{1}{\pi}\left(\int_{-\pi}^{0} -1\,dt + \int_0^{\pi} 1\,dt\right) = 0.
\]

\subsection{Step 2: Compute $a_n$}
\[
a_n = \frac{1}{\pi}\int_{-\pi}^{\pi} f(t)\cos(nt)\,dt.
\]
Since $f(t)$ is odd and $\cos(nt)$ is even, their product is odd. Thus
\[
a_n = 0.
\]

\subsection{Step 3: Compute $b_n$}
\[
b_n = \frac{1}{\pi}\int_{-\pi}^{\pi} f(t)\sin(nt)\,dt.
\]
Now $f(t)$ is odd and $\sin(nt)$ is odd $\implies$ product is even:
\[
b_n = \frac{2}{\pi}\int_{0}^{\pi} (1)\sin(nt)\,dt.
\]
\[
b_n = \frac{2}{\pi} \left[ \frac{1 - \cos(n\pi)}{n} \right] = \frac{2}{n\pi}(1 - (-1)^n).
\]

So:
\[
b_n =
\begin{cases}
\frac{4}{n\pi}, & n \text{ odd}, \\
0, & n \text{ even}.
\end{cases}
\]

\subsection{Final Fourier Series (Sine Form)}
\[
f(t) = \frac{4}{\pi}\left( \sin t + \frac{1}{3}\sin 3t + \frac{1}{5}\sin 5t + \cdots \right).
\]

\subsection{Complex Form of Fourier Series}
We can also write the Fourier series using complex exponentials:
\[
f(t) = \sum_{n=-\infty}^{\infty} c_n e^{int}, \quad
c_n = \frac{1}{2\pi} \int_{-\pi}^{\pi} f(t) e^{-int}\,dt.
\]

\subsection{Step 1: Compute $c_n$}
For the square wave:
\[
c_n = \frac{1}{2\pi}\left(\int_0^\pi 1 \cdot e^{-int}\,dt + \int_{-\pi}^0 -1 \cdot e^{-int}\,dt\right).
\]

\[
c_n = \frac{1}{2\pi}\left(\frac{1-e^{-in\pi}}{-in} - \frac{e^{in\pi}-1}{-in}\right).
\]

\[
c_n = \frac{1}{\pi in}(1-(-1)^n).
\]

Thus:
\[
c_n =
\begin{cases}
\frac{2}{i\pi n}, & n \text{ odd}, \\
0, & n \text{ even}.
\end{cases}
\]

\subsection{Step 2: Final Complex Series}
\[
f(t) = \sum_{\substack{n=-\infty \\ n\ \text{odd}}}^{\infty} \frac{2}{i\pi n} e^{int}.
\]

\subsection{Interpretation}
\begin{itemize}
    \item The sine form shows explicitly the harmonic content (only odd harmonics).
    \item The complex form is more compact and symmetric, useful in analysis and physics.
    \item Both represent the same square wave in the limit of infinite harmonics.
\end{itemize}

\section{\Huge\textbf{Fourier Transforms}}
The Fourier Transform (FT) converts a time-domain signal into its frequency-domain representation.

\subsection{From Fourier Series to Fourier Transform (Step by Step)}

A periodic signal $x_T(t)$ of period $T$ can be expressed as a Fourier Series:
\[
x_T(t) = \sum_{n=-\infty}^{\infty} c_n e^{i n \omega_0 t}, 
\qquad \omega_0 = \frac{2\pi}{T}.
\]

The coefficients are
\[
c_n = \frac{1}{T} \int_{-T/2}^{T/2} x_T(t)\, e^{-i n \omega_0 t}\,dt.
\]

\begin{enumerate}
    \item As $T \to \infty$, $x_T(t) \to x(t)$ (aperiodic signal).
    \item The spacing between harmonics $\omega_0 = 2\pi/T \to 0$, giving a continuous frequency axis $\omega$.
    \item The discrete coefficients $c_n$ become continuous spectrum values $X(\omega)$.
    \item The summation turns into an integral.
\end{enumerate}

Thus, the Fourier Transform pair is obtained:
\[
X(\omega) = \int_{-\infty}^{\infty} x(t)\, e^{-i \omega t}\,dt, 
\qquad
x(t) = \frac{1}{2\pi} \int_{-\infty}^{\infty} X(\omega)\, e^{i \omega t}\,d\omega.
\]

\subsection{Time and Frequency Domains}

\begin{itemize}
    \item \textbf{Time domain $x(t)$:} signal evolution with respect to time.
    \item \textbf{Frequency domain $X(\omega)$:} how much of each sinusoidal frequency is present.
\end{itemize}

\begin{center}
\begin{tikzpicture}[scale=1]
  % Time domain pulse
  \draw[->] (-3,0) -- (3,0) node[right] {$t$};
  \draw[->] (0,0) -- (0,2.2);
  \draw[thick] (-1,0) -- (-1,1.5) -- (1,1.5) -- (1,0);
  \node[above] at (0,1.5) {$x(t)$};

  % Arrow
  \draw[->, thick] (3.5,1) -- (5.5,1) node[midway,above] {Fourier Transform};

  % Frequency domain spectrum
  \begin{scope}[shift={(8,0)}]
    \draw[->] (-3,0) -- (3,0) node[right] {$\omega$};
    \draw[->] (0,0) -- (0,2.2);
    \draw[thick, domain=-2.8:2.8, smooth, samples=60] plot (\x,{1.5*(sin(deg(2*\x))/(2*\x))});
    \node[above] at (0,1.5) {$X(\omega)$};
  \end{scope}
\end{tikzpicture}
\end{center}

A localised rectangular pulse in time spreads into a sinc in frequency. Conversely, a single sinusoid in time becomes a delta spike in frequency.

\subsection{Fourier Transform Properties: Shifts}

\begin{itemize}
    \item \textbf{Time Shift Theorem:}  
    If $x(t) \leftrightarrow X(\omega)$, then
    \[
    x(t-t_0) \quad \Longleftrightarrow \quad e^{-i \omega t_0} X(\omega).
    \]
    (Time delay $\to$ frequency phase shift)

    \item \textbf{Frequency Shift Theorem:}  
    If $x(t) \leftrightarrow X(\omega)$, then
    \[
    e^{i \omega_0 t} x(t) \quad \Longleftrightarrow \quad X(\omega - \omega_0).
    \]
    (Modulation in time $\to$ spectrum shift)
\end{itemize}

\begin{center}
\begin{tikzpicture}[scale=0.9]
  % Time shift
  \draw[->] (-3,0) -- (3,0) node[right] {$t$};
  \draw[->] (0,0) -- (0,2);
  \draw[thick] (0.5,0) -- (0.5,1.2) -- (2,1.2) -- (2,0);
  \node[above] at (1.25,1.2) {$x(t-t_0)$};

  \draw[->, thick] (3.5,1) -- (5.5,1);

  \begin{scope}[shift={(8,0)}]
    \draw[->] (-3,0) -- (3,0) node[right] {$\omega$};
    \draw[->] (0,0) -- (0,2);
    \draw[thick, domain=-2.5:2.5, smooth, samples=50] plot (\x,{1.2*(sin(deg(2*\x))/(2*\x))});
    \node[above] at (0,1.2) {$e^{-i\omega t_0} X(\omega)$};
  \end{scope}
\end{tikzpicture}
\end{center}

\subsection{Worked Example: Rectangular Pulse}

Let
\[
x(t) =
\begin{cases}
1, & |t| \leq \tfrac{T}{2}, \\
0, & |t| > \tfrac{T}{2}.
\end{cases}
\]

Fourier Transform:
\[
X(\omega) = \int_{-T/2}^{T/2} e^{-i \omega t}\,dt
= \frac{2\sin(\omega T/2)}{\omega}.
\]

Thus
\[
X(\omega) = T \cdot \mathrm{sinc}\!\left(\frac{\omega T}{2\pi}\right).
\]

\section{\Huge\textbf{Derivation of the Discrete Fourier Transform (DFT)}}

\subsection{Step 1: Continuous-Time Fourier Transform (CTFT)}

For a continuous-time signal $x(t)$, the Fourier Transform is
\[
X(\omega) = \int_{-\infty}^{\infty} x(t) e^{-i\omega t}\,dt.
\]

\subsection{Step 2: Sampling the Signal}

We sample $x(t)$ every $T_s$ seconds:
\[
x[n] = x(nT_s), \quad n \in \mathbb{Z}.
\]

This produces a discrete-time sequence.

\subsection{Step 3: Discrete-Time Fourier Transform (DTFT)}

The DTFT is
\[
X(e^{i\omega}) = \sum_{n=-\infty}^{\infty} x[n]\, e^{-i \omega n}.
\]

Properties:
\begin{itemize}
    \item $X(e^{i\omega})$ is continuous in $\omega$.
    \item $X(e^{i\omega})$ is $2\pi$-periodic.
\end{itemize}

\subsection{Step 4: Finite-Length Signals}

In practice, we only keep $N$ samples:
\[
x[n] = 0 \quad \text{for } n < 0 \text{ or } n \geq N.
\]

Equivalently, we treat $x[n]$ as $N$-periodic:
\[
x[n+N] = x[n].
\]


\end{document}